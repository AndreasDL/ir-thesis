\chapter{A first look at the data}
{\samenvatting }

\section{The DEAP dataset}
This thesis uses the DEAP dataset\cite{DEAP}, a dataset for emotion analysis that is publicly available for academic research. This dataset contains EEG recordings of 32 participants, each watching 40 one minute excerpts of music videos. Each video was rated individually by each person on valence, arousal, dominance and liking. The first three ratings correspond to the valence, arousal and dominance space of an emotion \ref{valarrdomspace}. The liking component indicates how much the person liked the video excerpt and should not be confused with the valence component; it inquires information about the participants' tastes, not their feelings, i.e. a person can like a video that triggers angry or sad emotions. The liking rates are neglected, since they are not part of the emotion space.

\npar

For assessment of these scales, the self-assessment manikins (SAM)\nomenclature{SAM}{self-assessment manikins}, were used\cite{DEAP}. SAM visualizes the valence, arousal and dominance scale with pictures, each picture corresponds to a discrete value. The user can click anywhere in between the different figures, which makes the scales continuous. All dimension are given by a float between 1 and 9, but for the context of this thesis, a preprocessing step scaled and translated these values to ensure they range between 0 and 1.

\npar

The used SAM figures are shown in Figure \ref{SAM}. The first row gives the valence scale, ranging from sad to happy. The second row shows the arousal scale, ranging from bored to excited. The last row represents the different dominance levels. The left figure represents a submissive emotion, while the right figure corresponds with a dominant feeling.

\mijnfiguur{width=0.5\textwidth}{SAM}{The images used for the SAM\cite{DEAP}.}

%verdeling waardes
To further inspect the distribution of the user ratings and whether or not the data is balanced, the average for each emotion dimension (valence, arousal and dominance), was determined using all videos of all persons. These can be seen in Table \ref{avg-vals}. A uniform distribution, which is the ideal case for machine learning, should give a value of 0.5. The averages of the different dimensions are a just a little above 0.5, which gives a first indication that the data is only slightly unbalanced.

\begin{table}[H]
\centering
\begin{tabular}{l|lll}
\textbf{}  & \textbf{Valence} & \textbf{Arousal} & \textbf{Dominance} \\ \hline
\textbf{value} & 0.532     	  & 0.520  			 & 0.548
\end{tabular}
\caption{The average value of each component.\label{avg-vals}}
\end{table}

To further inspect the data, each component was divided in 8 ranges or bins. Next all user ratings are placed in their corresponding bin. Then the percentage of movies for each bin was calculated. This shown in Figure \ref{valence}, \ref{arousal} and \ref{dominance} for valence, arousal and dominance respectively. It is clear that even though all components have most of their weight in the higher bins, the data overall is more or less balanced.   

\mijnfiguur{width=0.7\textwidth}{valence}{The distribution of the valence values.}
\mijnfiguur{width=0.7\textwidth}{arousal}{The distribution of the arousal values.}
\mijnfiguur{width=0.7\textwidth}{dominance}{The distribution of the dominance values.}

\section{a }
