%\subsubsection{Event related potentials}
%
%An Event related potential (ERP)\nomenclature{ERP}{Event Related Potential}, is a measured brain response to an event, measured by EEG or MEG. The ERP nomenclature usually starts with a letter that indicates the polarity: the P corresponds to a positive polarity, while the N indicates a negative polarity. The number indicates the mean latency, measured between the ERP and the stimulus, which might may variate significantly between subjects.
%
%\npar
%
%The most important one is the P300 wave, which is usually elicited using the oddball paradigm. The oddball paradigm is the occurence of a low probability target item between high probability targets, e.g. flashing a specific symbol in a grid of different symbols. It consists of two components, the P3a with a latency of 240ms and the P3b with a latency of 350 ms\cite{P300TwoParts}. The later component, P3b only occurs when the subject actively counted either the targeted or more frequent stimuli.
%
%\mijnfiguur{width=0.45\textwidth}{waves}{The Different ERP linked to an oddball paradigm, found at\cite{P300Figure}.}
%
%\subsection{The P300 speller}
%
%The P300 speller is an active topic of research that uses EEG data to enable persons with the locked in syndrome to communicate\cite{P300Origin}. The basic version uses a six by six grid of characters, each row and column is flashed in a random order while the subject silently counts the number of flashes of a certain character, as shown in figure \ref{P300SpellerPerson}. This procedure, where a train of stimuli with some infrequent occurring target stimuli is applied, is called the oddball paradigm\cite{PaperThibault}. It is known that this technique triggers an increase in the potential difference in the EEG around the parietal lobe. This ERP occurs +- 300 milliseconds after the stimuli is flashed, hence its name, the P300 waveform\cite{ComparisonClassifications}. The presence or absence of the P300 waveform is used by the P300 speller to determine what character the subject was focusing on, which basically allows the subject to spell text. 
%
%\mijnfiguur{width=0.7\textwidth}{P300SpellerPerson}{Different parts of the P300 speller, found at \cite{P300SpellerPerson}.}
%
%To improve the spelling time, many improvements and research has been done. Language models were used to predict the word based on the first characters, which enabled great speedups\cite{LangModel}, classifiers were compared and tested on both healthy\cite{ClassTechniqueComp} and unhealthy subjects\cite{ComparisonClassifications}. Since many unhealthy subjects might have an impaired vision or eye movement, tactile\cite{TactileP300} and auditory\cite{AuditoryP300} spellers have been developed to circumvent this problem.
%
%\npar
%
%To improve accuracy, common problems such as adjacency distraction, when a subject is distracted by a neighboring flash, and double flashes, when the target row and column are flashed close after each other, were avoided using new randomized paradigms\cite{PaperThibault}. Other input layouts like the T9 interface P300 speller have also been developed\cite{P300T9}. 
%
%\npar
%
%To further speedup the spelling, error potentials were explored. Error potentials are triggered when the user becomes aware of an erroneous action\cite{ErrorPotentials}, i.e. when a wrong character is selected. When an Error potential is detected, the character is usually changed to the second most probable character according to the P300 decoding\cite{ErrSecChar}, which is the most viable character.
%
%\npar
%
%The basic P300 speller needs a calibration period before it can be used, when a healthy subject makes a mistake during calibration, he can simply communicate this. This is not the case for a patient, who has no other means of communication than the P300 speller. Having wrongly labeled data during calibration can lead to severe problems. The unsupervised speller as proposed in \cite{P300Unsupervised} solves this problem by removing the need for a calibration procedure. The speller works with expectation maximization and has an undemanding linear classification backend. This system starts with a warm-up period where the system adapts to the given condition.


%\subsection{Benefits of creating an emotionally aware P300 speller}
%
%Emotions play a major role in non-verbal communication, are quite complex and essential to understand human behavior. The ability to recognize emotions will improve the ability of computers to understand human interaction\cite{CompRecognizeEmotion} and are likely to improve the P300 speller's accuracy in different ways.
%
%\npar
%
%To improve the detection of error potentials, emotion can be used. It is expected that when a person makes a lot of mistakes, his/her emotional state will change to a less happy, more frustrated state. Making the speller emotionally aware, could improve the detection of error potentials.
%
%\npar
%
%Research with visual stimuli on healthy subjects, show that emotion has an effect on the auditory P300 wave\cite{AuditoryP300Effect}. Both the P300 peak amplitude and area was highest when viewing neutral pictures and descended further, in decreasing order, for sadness, anger and pleasure. The amplitudes were significantly lower at both Fz and C3 positions than Pz and Oz. The latency of the P300 ERP speller was shortest or neutrality and in increasing order longer for pleasure, anger and sadness. It is expected that a visually triggered P300 wave, will also be influenced by emotion. Detecting emotion can therefore improve the detection of P300 waves.
%
%\npar
%
%%Contrary to what subjects might think, the P300 speller is unable to read the mind and know what a person is thinking about\cite{P300Origin}. The P300 speller provides no more than a means of communication that the subject can use. Should he chose to ignore the instructions and focus his attention elsewhere, then the recordings become useless. Nevertheless, ethical question often remain unanswered. Knowing how the subject feels, can help him communicate more humane on one hand, while providing more insight for ethical issues, on the other hand, e.g. "How does the subject think about the P300 speller recording and analyzing his brain activity?". Information about the subject's emotional state can help answer some of these ethical questions.
%
%
%\section{Goal of the thesis}
%
%This thesis aims to improve the performance of the P300 speller by making the speller emotionally aware. An emotional aware speller is expected to yield better performance, since the P300 wave is affected by the emotion. Furthermore, the detection of error potentials can be combined with the emotional state, since the emotional state is expected to change with increasing errors.
%
%\npar
%
%More concrete, the main goal is to recognize emotions in the arousal-valence model, using the DEAP dataset. First the emotions of a single person should be recognized, since the features are known to differ from person to person. Later, the model will be adjusted so that it is capable of detecting emotion of different persons. Once the emotion recognition is able to classify the emotions with decent accuracy, it will be integrated in the P300 speller, which should give additional accuracy and aid in the error potential detection. The expected results are:
%\begin{itemize}
%\item Being able to recognize emotions of a single person
%\item Being able to recognize emotions across different persons
%\item Improved accuracy for the P300 speller
%\item Improved error potential recognition for the P300 speller
%\end{itemize}
%
%\npar
%
%Additionally the gained information for emotion recognition can be used for other ethical research studies, to answer ethical questions about the usage of BCI on patients.