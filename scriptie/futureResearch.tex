\chapter{Future Research}
{\samenvatting This chapter gives an overview of research that can be done on this subject}

\section{Applications for emotion recognition}

\subsection{Emotionally aware P300 speller}
%TODO marketing apart

Emotion recognition has many different applications, e.g. as an improvement for the P300 speller or marketing analysis. The P300 speller is a very well-known, academic application of BCI and an active topic of research. It uses EEG data to enable patients with a locked in syndrome to communicate\cite{P300Origin}. The basic version uses a six by six grid of characters, each row and column is flashed in a random order while the subject silently counts the number of flashes of a certain character, as shown in figure \ref{P300SpellerPerson}. This procedure, where a train of stimuli with some infrequent occurring target stimuli is applied, is called the oddball paradigm\cite{PaperThibault}. It is known that this technique triggers an increase in the potential difference in the EEG around the parietal lobe. When a potential difference in the brain occurs as a reaction to an event, it is referred to as en event-related potential (ERP) \nomenclature{ERP}{Event-Related Potential}. The P300 ERP occurs roughly 300 milliseconds after the stimulus is flashed, hence its name\citep{ComparisonClassifications}. The presence or absence of the P300 waveform is used by the P300 speller to determine what character the subject was focusing on, which basically allows the subject to spell text. 

\mijnfiguur{width=0.7\textwidth}{P300SpellerPerson}{Different parts of the P300 speller, found at \cite{P300SpellerPerson}.}

Research with visual stimuli on healthy subjects, has shown that emotion has an effect on the auditory P300 wave\cite{AuditoryP300Effect}. Both the P300 peak amplitude and area were highest when viewing neutral pictures and descended further, in decreasing order, for sadness, anger and pleasure. The latency of the P300 ERP speller was shortest for subjects in an emotionally neutral state. The latency increased for pleasure, anger and sadness. It is expected that a visually triggered P300 wave, will also be influenced by emotion. Having a good emotion recognition system, can help a P300 detector in finding the correct latency of the P300 wave. This can then, in turn improve the detection of P300 waves. Additionally knowing a subject's emotional state can help detecting when a subject gets frustrated, e.g. because of mistakes he makes.

\npar

An improvement in performance is not the only advantage an emotionally aware P300 speller has. Contrary to what subjects might think, the P300 speller is unable to read the mind and know what a person is thinking about\cite{P300Origin}. The P300 speller provides no more than a means of communication that the subject can use. Should he chose to ignore the instructions and focus his attention elsewhere, then the recordings become useless. Nevertheless, ethical questions often remain unanswered. Knowing how the subject feels, can provide more insight for ethical issues, e.g. "How does the subject think about the P300 speller recording and analysing his brain activity?". Information about the subject's emotional state can help answering some of these ethical questions. Integrating the results from this thesis with the P300 speller, is an opportunity for future research.

\section{Marketing analysis}
Another application for emotion recognition is in the field of marketing and customer satisfaction research. Discovering how a person feels about a product is often tricky. Questionnaires is one way to go, but they might contain a lot of noise. Being able to 'read' the emotion straight from a subject's mind, is expected to give more accurate results as it avoids any form of social masking. %TODO ref
