\chapter{Methods}
{\samenvatting This section will first explain what feature selection means and why it is used, before giving a detailed overview of the different methods.}

\section{Similar research}
This section will give a overview of (some) other studies that did similar research and their conclusions.

\subsection{DEAP method}
The first method of emotion recognition is the DEAP method, described in the DEAP paper \citep{DEAP}.


\section{Feature selection}
Feature selection is the process of selecting good features from a set of feature. The need for this is twofold: first by reducing the number of features, you can protect yourself against overfitting. This is important when the dataset is limited. Second, reducing the number of features can speedup the learning process of a learning algorithm as fewer parameters need to be optimized. Additionally, in the context of research and this thesis, looking at which features are important gives insight in the problem. In the context of this thesis, knowing what feature are relevant can help neuroscientists understand the working of the brain better.

\subsection{Pearson Correlation}
The Pearson correlation coefficient measures the linear relationship between two variables. The output is a value r, that lies between -1 and 1, corresponding to perfect negative correlation and perfect positive correlation respectively. A correlation value of 0 means that there is no correlation.

\npar

More formally\citep{corrPaper}, the Pearson product-moment coefficient of correlation, r between variables $X_i$ and $Y_i$ of datasets $X$ and $Y$ is defined as:


\begin{center}
$r = \frac{SS_{xy}}{\sqrt{SS_{xx}SS_{yy}}}$
\end{center}
with
\begin{center}
$SS_{xy} = \sum\limits_i (X_i-\tilde{X})(Y_i-\tilde{Y})$
\end{center}
and
\begin{center}
$SS_{xx} = \sum\limits_i (X_i-\tilde{X})^2$ \\
$SS_{yy} = \sum\limits_i (y_i-\tilde{Y})^2$
\end{center}

\npar

The Pearson correlation coefficient is fast and simple to calculate, but has some major shortcomings. First off, it can only see linear relation ships and will not see the correlation between a value $x$ and $x^2$.

\npar

In the context of this thesis, whether the correlation is positive or negative is not important; a learning algorithm needs features that have significant correlation. As a result the absolute value of the r value is reported as this allows for faster comparison of correlations.

\subsection{Mutual Information}
Mutual information is a more robust option for correlation estimation. The mutual information, I, of two variables $X$ and $Y$ is defined as \citep{mutPaper}:
\begin{center}
$I(X,Y) = \sum\limits_{y\in Y} \sum\limits_{x\in X} p(x,y)log(\frac{p(x,y)}{^(x)p(y)}$
\end{center}

\npar

Using the mutual information directly for feature ranking might be inconvenient for two reasons. Firstly, it doesn't lie in a fixed range and it is hard to compute for continuous variables. One solution for this problem is to normalize the mutual information scores, so that the results lies between 0 and 1.

The normalized mutual information, NMI of variables X and Y is given by:
\begin{center}
$NMI(X,Y) = \frac{H(X) + H(Y)}{H(X,Y)}$
\end{center}
With $H(X)$ and $H(Y)$ being the Shannon entropy of variable X and variable Y, defined as:
\begin{center}
$H(X) = \sum\limits_{i\in X} p_ilog(\frac{1}{p_i}) = - \sum\limits_i p_ilog(p_i)$\\
$H(Y) = \sum\limits_{i\in Y} p_ilog(\frac{1}{p_i}) = - \sum\limits_i p_ilog(p_i)$
\end{center}
\npar

%\subsection{ANOVA}
%TODO

\subsection{Distance Correlation}
Distance correlation is a relatively new technique that is designed explicitly to address shortcomings of Pearson correlation. A Pearson correlation coefficient of zero implies that the variables might be independent, but as mentioned before, does not guarantee this. 

\npar

The distance covariance is defined as\citep{distPaper}:
\begin{center}
$dCov^2(X,Y) = \frac{1}{n^2}\sum_limits_{k,l=1}^{n} A_{k,l}B_{k,l}$
\end{center}
With A, B being simple linear functions of the pairwise distances between sample elements. This metric is a covariance metric, which means that it is not normalized. The distance correlation is the normalized version of the distance covariance and is defined as:

\begin{center}
$dCor(X,Y) = \frac{dCov(X,Y)}{\sqrt{dVar(X)dVar(Y)}}$
\end{center}
With $dCov(X,Y)$ being the aforementioned distance covariance, $dVar(X)$ and $dVar(Y)$ are the distance standard deviations. 

\npar

The distance correlation has the disadvantage that is much slower than mutual information or Pearson correlation, but in return, the distance correlation is able to detect more complex relationships between two variables.

\subsection{Linear Regression}
Another way of finding relevant features is to use model based ranking. In model based ranking an arbitrary machine learning method is used to build a model. Looking at the coefficients of the trained model, the importance is determined by its own coefficient. High coefficients mean that the feature has a lot of influence on the output, while low coefficients correspond to less important features.

\npar

The first method to use is simple linear regression. This method tries to find a linear combination of features that produces the output value. Linear regression can achieve good results given that the data doesn't contain a lot of noise and the features are (relatively) independent. When the set of features contains correlated features, the model becomes unstable. As a result, small changes in input data might lead to huge differences in output coefficients. for example assume the 'real output' is given by $Y = X_1 + X_2$ and the dataset contains output in the form of $Y = X_1 + X_2 + \epsilon$ with $\epsilon$ being some random noise. Further more assume that $X_1$ and $X_2$ are linearly correlated, meaning that $X_1 \approx X_2$. The suspected output of the model should be $Y = X_1 + X_2$, but since noise is added the algorithm might end up with arbitrary combinations of $X_1$ and $X_2$, e.g. $Y = -X_1 + 3X_2$ and rate one feature much higher than another one, while in reality they are of equal importance. This is due to the noise; by maximizing the performance, the algorithm will minimize the influence of noise on the output, which result in unstable behaviour when sufficient correlated features are present. 

\clearpage

\subsection{Lasso Regression}
Lasso regression uses L1 regularization, that adds a penalty $\alpha\sum\limits_{i=1}^{n} |w_i|$ to the loss function. the result is that the coefficients of weak features are forced to zero, as each non-zero feature adds to the penalty. This form is regularization is thus quite aggressive, it removes weak features completely. The problem with this is, again, stability; coefficient can vary significantly even for small changes in training data, when there are correlated features.

\subsection{Ridge Regression}
Ridge regression uses L2 regularization, which add a L2 norm penalty to the loss function, given by $\alpha\sum\limits_{i=1}^{n} w_i^2$. Where the L1 norm forces the coefficients to zero, the L2 regularization forces the coefficients to be spread out more equally. The result is that correlated features tend to get similar coefficients, as this minimizes the loss function, which in turn results in a more stable model. 

\subsection{SVM}
Just like Regression uses linear regression to get coefficients, it is also possible to use SVM for feature importance estimation.

\subsection{Random Forests}
Random forests (RF) \nomenclature{RF}{Random Forests} is a efficient learning algorithm based on model bagging and aggregation ideas\citep{rfPaper}. The Random forests work by creating different decision trees. On their own, decision trees are very prone to overfitting. Random forests solve this problem by creating an aggregation of trees. 

\npar

Additionally, some randomness is included, each tree looks at a random subset of the samples and a random subset of the features. This principle is shown in Figure \ref{RF}. This random subset of samples is called the bootstrap sample and is selected out of N samples, by picking N times a sample, with replacement. This results, on average, in 2/3 of the samples being selected (with some doubles). The other 1/3 of the samples are then used as out of bag (oob) \nomenclature{oob}{out of bag} set. Averaging the performance of each tree on the out of bag set, offers an indication of the generalisation of the random forest.

\mijnfiguur{width=0.9\textwidth}{RF}{The structure of a random forest, found at \citep{rfPic}}

\npar

To understand which features are good, one needs to understand the internal workings of a decision tree. Suppose the following example\footnote{This example is based extensively on this youtube video: https://www.youtube.com/watch?v=eKD5gxPPeY0}, where one tries to find an algorithm to predicted whether or not a person will play tennis on a given day. Suppose the training data is given by Table \ref{decisionTreeTable} and a prediction for the $15^{th}$ sample needs to be made.

\begin{table}[H]
\centering
\caption{suppose the following training examples for a decision tree.}
\label{decisionTreeTable}
\begin{tabular}{lllll}
\textbf{Day} & \textbf{Outlook} & \textbf{Humidity} & \textbf{Wind} & \textbf{Play tennis} \\
\textbf{1}   & sunny            & high              & weak          & no                   \\
\textbf{2}   & sunny            & high              & strong        & no                   \\
\textbf{3}   & overcast         & high              & weak          & yes                  \\
\textbf{4}   & rain             & high              & weak          & yes                  \\
\textbf{5}   & rain             & normal            & weak          & yes                  \\
\textbf{6}   & rain             & normal            & strong        & no                   \\
\textbf{7}   & overcast         & normal            & strong        & yes                  \\
\textbf{8}   & sunny            & high              & weak          & no                   \\
\textbf{9}   & sunny            & normal            & weak          & yes                  \\
\textbf{10}  & rain             & normal            & weak          & yes                  \\
\textbf{11}  & sunny            & normal            & strong        & yes                  \\
\textbf{12}  & overcast         & high              & strong        & yes                  \\
\textbf{13}  & overcast         & normal            & weak          & yes                  \\
\textbf{14}  & rain             & high              & strong        & no                   \\
             &                  &                   &               &                      \\
\textbf{15}  & rain             & high              & weak          & ?                   
\end{tabular}
\end{table}

\npar
A decision tree will take a feature and split the data based on the possible outcomes of this feature. In case the features are continuous values, ranges are selected. In some cases the leafs will be pure, like the leaves displayed in  green in Figure \ref{decisionTree}. All examples in here have the same output. In case the leave is not pure, then another split is needed. Note that not all random forests split until all leaves are pure; random forest can be limited in depth, in that case the output is chose by a majority voting of the samples.

\mijnfiguur{width=0.6\textwidth}{decisionTree}{A decision tree for the data in Table \ref{decisionTreeTable}}

Once the tree is constructed it becomes clear that the predicted output of sample 15 is yes. This is obtained simply by following the tree branches. Even though the features are selected at random, they have influence on the accuracy. Good features will reduce the impurity significantly, thus the impurity reductions are a good indication for how important a feature is.

\npar

Since the importance is averaged over different nodes and different trees, it is also capable of detecting combinations of features that work well. One feature may not be important on its own, but might be a very good feature when combined with other features. Suppose the following example in Table \ref{featPair}:

\begin{table}[H]
\centering
\label{featPair}
\begin{tabular}{lll}
\textbf{label} & \textbf{feature A} & \textbf{feature B} \\
\textbf{Happy} & +                  & +                  \\
\textbf{Happy} & -                  & -                  \\
\textbf{Sad}   & -                  & +                  \\
\textbf{Sad}   & +                  & -                 
\end{tabular}
\caption{Some feature are not significant on its own, but a might be part of a combination of features.}
\end{table}

It is clear that feature A and B are very important when it comes to predicting whether or not a person is happy or sad. When both features have the same sign, the person is happy, otherwise he is not. This problem occurs in many simple selection methods. 

\npar

This problem does not occur for random forest though, as combinations of features are also 'tested' in the sense that a tree might split on them in different stages. Once the combination of features occurs randomly in a decision tree, the impurity will drop significantly, which will result in higher importance rankings.

%svm
%rf
%PCA ?

%Rf werkt beter omdat het beter combinaties van features kan inschatten!

%Advanced RF method + paper

\section{Features}
\label{featuresExplained}
The next step is defining relevant physiological features for emotion recognition, two categories of features are observed: EEG features and non-EEG features.
The EEG features are features extracted from the electroencephalography measurements from the subject's scalp. This section will go through the used EEG features in this thesis.
\npar
The power spectral density (PSD) \nomenclature{PSD}{Power Spectral Density} of a signal gives the distribution of the signal's energy in the frequency domain. By calculating the spectral density for different waveband of the signal, one can determine how much alpha, beta, ... power is in the signal.
\npar
Differential entropy is defined as follows \citep{killyPaper} \\
\begin{center}
$ - \int_{-\infty}^{\infty} \frac{1}{\sqrt{2\pi\sigma^2}} exp(\frac{(x-\mu)^2}{2\sigma^2}) log(\frac{1}{\sqrt{2\pi\sigma^2}}) exp(\frac{(x-\mu)^2}{2\sigma^2})dx$
\end{center}
According to \citep{diffEnt}, the differential entropy of a certain band is equivalent to the logarithmic power spectral density for a fixed length EEG sequence, which simplifies the calculations significantly.
\npar
The most known feature for valence recognition is the frontal asymmetry of the alpha power\cite{GivenPaper}.
The right hemisphere is generally speaking, more active during negative emotion than the left hemisphere which is in turn more active during positive emotions\cite{RealTimeEEGEmotion,EEGDatasets,killyPaper}. The asymmetry can be calculated in different ways, one of them is the differential asymmetry (DASM) \nomenclature{DASM}{Differential Asymmetry}, where the left alpha power is subtracted from the right alpha power.
\begin{center}
$DASM = DE_{left} - DE_{right}$
\end{center}
Another way to measure the asymmetry if by division. The Rational Asymmetry (RASM) \nomenclature{RASM}{Rational Asymmetry} does exactly this and is given by: \\
\begin{center}
$RASM = \frac{DE_{left}}{DE_{right}}$
\end{center}
With $DE_{left}$ and $DE_{right}$ being the left and right differential entropy respectively.
\npar
Another reported feature in literature is the caudality, or the asymmetry in fronto-posterior direction\cite{caudality}. This can again be calculated in two ways. The first method is the differential Caudality (DCAU) \nomenclature{DCAU}{Differential Caudality} is defined as: \\
\begin{center}
$DCAU = DE_{front} - DE_{post}$
\end{center}
Another way to determine the Caudality is the Rational Caudality (RCAU) \nomenclature{RCAU}{Rational Caudality}, which is defined as:
\begin{center}
$RCAU = \frac{DE_{front}}{DE_{post}}$
\end{center}
With $DE_{front}$ and $DE_{post}$ being the frontal and posterior power respectively.
\npar
One way to determine the arousal is by looking at the different wavebands. Each waveband has their own medical interpretation, see \ref{wavebands}. More alpha power corresponds to a more relaxed brain, while more beta power corresponds to a more active brain. The alpha / beta ratio therefore seems a good indicator for the arousal state of a person.
\npar
The aforementioned EEG features are just one class of physiological features. The DEAP dataset contains several physiological measurements, listed below \citep{DEAP}. For each of these measurements the average, standard deviation, variation, median, minimum, maximum and the standard deviation is calculated.
\npar
The Galvanic Skin Response \nomenclature{GSR}{Galvanic Skin Response} uses two electrodes on the middle and index finger of the subjects left hand to measure the skin resistance. It has been reported that the mean value of the GSR is related to the level of arousal\citep{GSR, DEAP}.
\npar
The respiration belt, indices the user's respiration rate. Slow respiration is linked to relaxation (low arousal), while fast and irregular respiration patterns corresponds to anger or fear, both emotions with low valence and high arousal\citep{DEAP}.
\npar
A plethysmograph is a measurement of the volume of blood in the subject's left thumb. This can be interpreted as the the blood pressure. Blood pressure offers valuable insight into the emotional state of a person as it correlated with emotion; stress is known to increase blood pressure\citep{DEAP}.
\npar
The heart rate is not explicitly in the DEAP dataset, but can be extracted from the plethysmograph, by looking at local minima and maxima\citep{DEAP}. This is clearly visible when looking at the plethysmograph's output, shown in Figure \ref{before}.
\mijnfiguur{width=0.7\textwidth}{before}{The plethysmograph before smoothing.}
\npar
The heart rate extraction is done in two steps. First the plethysmograph signal is smoothed using a butter filter to avoid noise being selected as local optima. In the second stage the local optima are located, which is shown in Figure \ref{extrema}

\mijnfiguur{width=0.7\textwidth}{extrema}{The local optima in the plethysmograph.}

\npar

These optima correspond to a heart beat, therefore the time between two consecutive local minima or maxima corresponds to the time between two heart beats, known as the interbeat interval. Getting the average heart rate from the interbeat interval is straight forward.

The last physiological feature is the skin temperature of the subject.