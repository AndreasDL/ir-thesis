\chapter{Emotion in the brain}
{\samenvatting This chapter starts with explaining some BCI basics, before discussing some commonly used methods to recognize emotion in the brain.}

\npar
How does one define emotion? Psychology makes a clear distinction between physiological behavior and the conscious experience of an emotion, called expression\cite{ExtendedPaper}. The expression consists of many parts, including the facial expression, body language and voice concern. Unlike expression, the physiological aspect of an emotion, e.g. heart rate, skin conductance and pupil dilation, is much harder to control. To really know one's emotions, it seems, one has to research the physiological aspect of the emotion. One possibility for this is analysis of brain activity via Electroencephalography\cite{EEGDatasets}.

\section{Brain Computer Interface (BCI) basics}

\subsection{Electroencephalography}
Different technologies exist to analyse the brain, the most convenient method is via Electroencephalography (EEG)\nomenclature{EEG}{Electroencephalography}, since it is a non-invasive method. Non-invasive methods, in contrast to invasive methods require no surgery; they simply measure electrical activity using electrodes placed on the scalp.

\npar

Signals originating from the cortex, close to the skull, are most visible, but signals originating deeper in the brain cannot be observed directly. Even for signals originating close to the cortex, EEG is far from precise as the bone between the the cortex and electrodes distorts the signal. Additionally other artifacts like eye and muscle movement add a lot a noise to the signal, that has to be removed. Even though the noise is persistent and has very low spatial resolution, EEG data can provide significant insight into the electrical activity of the cortex while offering excellent temporal resolution\cite{GivenPaper}.

\npar

Note that EEG records electrical activity, other methods like magnetoencephalography (MEG)\nomenclature{MEG}{magnetoencephalography} measure brain activity using magnetic fields. Since MEG is more prone to noise from external magnetic signals, i.e. the earth's magnetic field and electromagnetic communication, shielding is required, making this method less preferable.

\subsection{Electrode positioning and types}
As mentioned before, electrodes are placed on the scalp to measure the electrical activity. To ensure that experiments are replicable, standards for locations of electrodes have been developed. One of these systems is the 10/20 system, an internationally recognized methods to describe location of scalp electrodes\cite{TenTwentyManual}. The numbers 10 and 20 refer to the distances between the electrodes, which are either 10\% or 20\% of the total front-back or left-right distance of the skull. Each site is identified with a letter that determines the lobe and hemisphere location.
\begin{itemize}
\item \textbf{F:} Frontal
\item \textbf{T:} Temporal
\item \textbf{C:} Central
\item \textbf{P:} Parietal
\item \textbf{O:} Occipital
\end{itemize}
Note that no central lobe exists, the C letter is only used for identification purposes. The letter z indicates that the electrode is placed on the central line. Even numbers are use for the right hemisphere, while odd numbers are used for the left hemisphere. A picture of a 23 channel 10/20 system is added below for clarification. Note that some experiments may use more channels than shown in figure \ref{1020ElectrodePlacementSystem}, but they all follow the same naming convention.

\mijnfiguur{width=0.9\textwidth}{1020ElectrodePlacementSystem}{The electrode placement system\cite{1020Site}.}

Two different types of EEG channels exist, monopolar and dipolar. A monopolar channel records the potential difference of a signal, compared to a neutral electrode, usually connected to an ear lobe of mastoid. A bipolar channel is obtained by subtracting two monopolar EEG signals, which improves SNR by removing shared artifacts\cite{MonoBiPolar}. 

\subsection{Event related potentials}

The electrical activity in a brain is caused when an incoming signal arrives in a neuron. This triggers some sodium ions to move inside the cell, which in turn, causes a voltage rise\cite{ExtendedPaper}. When this increase in voltage reaches a threshold, an action potential is triggered in the form of a wave of electrical discharge that travels to neighboring neurons, this activity is captured by the surface electrodes.

\npar

An Event related potential (ERP)\nomenclature{ERP}{Event Related Potential}, is a measured brain response to an event, measured by EEG or MEG. The ERP nomenclature usually starts with a letter that indicates the polarity: the P corresponds to a positive polarity, while the N indicates a negative polarity. The number indicates the mean latency, which might may variate significantly between subjects.

\npar

The most important one is the P300 wave, which is usually elicited using the oddball paradigm; when a low probability target item occurs between high probability items. It consists of two components, the P3a with a latency of 240ms and the P3b with a latency of 350 ms\cite{P300TwoParts}. The later component, P3b only occurs when the subject actively counted either the targeted or more frequent stimuli.

\mijnfiguur{width=0.75\textwidth}{waves}{The Different ERP linked to an oddball paradigm.}

\clearpage

\subsection{Brain waves}
Usually, brain waves are split into different bands, based on their frequency\cite{EmotionRelativePower,WavesSite}.
\begin{enumerate}
\item \textbf{Alpha:} 8-13Hz, indicate how relaxed and/or inactive the brain is.
\item \textbf{Beta:} 13-30HZ, indicate a more active and focused state of mind.
\item \textbf{Gamma:} 30-50Hz, relate to simultaneous processing of information from different brain areas.
\item \textbf{Delta:} 0-4hz, these waves are generated during dreamless sleep and meditation.
\item \textbf{theta:} 4-8Hz, occur during dreaming.
\end{enumerate}
To remove most muscle and eye artifacts have a frequency around 1.2Hz, and artificats caused by nearby power lines, with a frequency of around 50Hz\cite{ExtendedPaper}, usually a bandpass filter is applied to filter out frequencies below 4Hz and above 40-45Hz.

\section{Extracting emotion from EEG data}

\subsection{Arousal - valence (- dominance) classification}
Before emotions can be recognized, a classification model is needed. A common model to classify emotions is the bipolar arousal-valence model\cite{ExtendedPaper,RealTimeEEGEmotion}, that places emotions in a two dimensional space. The main advantage of using a multidimensional model, is that all emotions are modeled in its space, even when no particular discrete label can be used to define the current feeling. Figure \ref{ArousalValenceModel} shows the mapping of different emotions for this model. 

\npar

Since this thesis aims to recognize emotion and change in emotion when a subject performs several erroneous actions, the linear model is the best choice, since it enables to track the state continuously.

\mijnfiguur{width=0.45\textwidth}{ArousalValenceModel}{The arousal - valence model maps emotions in a two dimensional plane.}

\subsection{Determining arousal}

Arousal can be determined in several ways. A first methods for the arousal recognition uses only alpha and beta bands. the Alpha band corresponds with a relaxed state and is often connected to brain inactivity\cite{ExtendedPaper,RealTimeEEGEmotion}. Beta waves, on the other hand, are an indication that the brain is a more active state or has a higher level of arousal. Combining these two parameters gives the beta/alpha ratio as an indication of the arousal level. 

\npar

Other methods are based on the EEG coherence across the prefrontal and posterior beta oscillations, which is known to increase when high arousal images are viewed. Additionally, the gamma power is said to increase with arousal with a delay of 500ms\cite{GivenPaper}.

\subsection{Determining valence}
The most known and used feature is the frontal asymmetry of the alpha power\cite{GivenPaper}.
The right hemisphere is generally speaking, more active during negative emotion than the left hemisphere which is in turn more active during positive emotions\cite{RealTimeEEGEmotion,EEGDatasets}. The assymmetry is given for L and R being the Left and Right frontal alpha powers as:\\
\begin{center}
$Index = \frac{L-R}{L+R}$
\end{center}
Computing the spectral power of the alpha band is possible via e.g. the fast fourier transform or wavelet transform.

\npar

It is also possible to include beta waves in the process. High alpha rates correspond with an inactive brain, while high beta waves with an active brain. Looking for an increase in beta activity and a decrease in alpha activity at one side, while the other hemisphere should show an increase in alpha waves and a decrease of beta waves as indication that is becomes less active, offers an insight to the frontal asymmetry and thus the valence\cite{ExtendedPaper}. 

\npar

Another features is the frontal midline (Fm)\nomenclature{Fm}{Frontal Midline} theta power, that is shown to increase with increasing pleasant ratings for audio stimuli\cite{MusicPaper}. %Gamma phase synchronization increases for unpleasant images.

\subsection{Datasets}
One of the most used datasets is the Dataset for Emotion Analysis using Physiological Signals (DEAP)\nomenclature{DEAP}{Dataset for Emotion Analysis using Physiological Signals} dataset\cite{DEAP}. This dataset contains EEG samples at 512 Hz of 32 persons each viewing 40 videos. A preprocessed version of this dataset, that is down sampled and has EOG removal was used extensively during this thesis.

\clearpage

Furthermore, the center for the study of emotion and attention (CSEA)\nomenclature{CSEA}{Center for the Study of Emotion and Attention}, by the university of Florida made several visual datasets available:
\begin{itemize}
\item \textbf{IAPS:} International Affective Picture System\nomenclature{IAPS}{International Affective Picture System}\\
This dataset consists of a large set op emotional stimuli in the form of color photographs.
\item \textbf{IADS:} International Affective Digital Sounds \nomenclature{IADS}{International Affective Digital Sounds}\\
This dataset consists of acoustic emotional stimuli designed for investigation and research of emotion and attention.
\item \textbf{ANEW:} Affective Norms for English Words\nomenclature{ANEW}{Affective Norms for English Words}\\
This dataset provides a set of emotional ratings for a large number of english words.
\item \textbf{ANET:} Affective Norms for English Text\nomenclature{ANET}{Affective Norms for English Text}\\
This set provides normative ratings of emotion for a large set of brief texts.
\item \textbf{SAM:} Self-Assessment Manikin\nomenclature{SAM}{Self-Assessment Manikin}\\
A non-verbal pictorial assessment technique that measures the pleasure, arousal and dominance associated with a person's affective reaction to a wide variety of stimuli.
\end{itemize}
The stimuli from these sets are often used in experiments to trigger emotions\cite{ExtendedPaper, EmotionSite, RealTimeEEGEmotion, EEGDatasets}.