%  Overzichtsbladzijde met samenvatting

\newpage
\addcontentsline{toc}{chapter}{Overview}
{
\setlength{\baselineskip}{14pt}
\setlength{\parindent}{0pt}
\setlength{\parskip}{8pt}

\begin{center}

\textbf{\huge
A Comparative Study of \\
Physiological Feature Selection Methods \\
for Emotion Recognition\\
}

by

Andreas DE LILLE

\end{center}

Supervisors: Prof.~J.~DAMBRE and Dr.~Ir.~P.~BUTENEERS \\
Counsellor: Ir.~T.~VERHOEVEN

Master's dissertation submitted in order to obtain the academic degree of\\
Master of Science in Computer Science Engineering

Department of Electronics and Information Systems\\
Chair: Prof. dr. ir. Rik Van de Walle\\
Faculty of Engineering and Architecture\\
Academic year 2015-2016\\



\section*{Summary}

An emerging field of research is the field of emotion recognition. Emotion can be observed in many ways, but the most reliable method is to use physiological signals. This method uses machine learning to classify emotions based on characteristics or features extracted from the signals. To do so, good, reliable features are needed. This work compares a wide range of features and feature selection techniques to study the physiological responses triggered by emotion.
\npar
Physiological signals from the brain dominate peripheral physiological signals completely. The results are that for a person specific emotion recognition system, EEG asymmetry features should be used with a focus on the frontal channels. In a cross-subject emotion recognition system, the difference between physiological brain signals and peripheral physiological signals is smaller.

\section*{Keywords}

Emotion recognition, Physiological Signals, Machine Learning, Feature Selection, EEG, valence, arousal, classification
}

\newpage % strikt noodzakelijk om een header op deze blz. te vermijden
