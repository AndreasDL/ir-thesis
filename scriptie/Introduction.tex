\chapter{Introduction}
{\samenvatting This chapter gives a quick abstract of the thesis, starting with explaining the P300 speller and P300 paradigm, before justifying the need for an emotionally aware P300 speller.}

%\begin{chapquote}{R.W. Picard, J. Klein\cite{CompRecognizeEmotion}}
%Bad feelings do not necessarily cause bad things to happen: mild irritations that simmer and create distress %or frustration, then boil into anger, can serve as impetus to find a better way.
%\end{chapquote}

\section{The P300 speller}
\subsection{The P300 speller explained}
The P300 speller is an active topic of research that utulises EEG data to enables persons with locked in syndrome to communicate\cite{P300Origin}. The basic version uses a six by six grid of characters, each row and column is flashed in a random order while the subject silently counts the number of flashes of a certain character, as shown in figure \ref{P300SpellerPerson}. This procedure, where a train of stimuli with some infrequent occurring target stimuli is applied, is called the oddball paradigm\cite{PaperThibault}. It is known that this technique triggers an increase in potential difference in the EEG around the parietal lobe. This Event Related Potential (ERP) occurs +- 300 milliseconds after the stimuli is flashed, hence its name, the P300 waveform\cite{ComparisonClassifications}. The presence or absence of the P300 waveform is used by the P300 speller to determine what character the subject was focusing on, which basically allows the subject to spell text. 

\mijnfiguur{width=0.7\textwidth}{P300SpellerPerson}{Different parts of the P300 speller, found at \cite{P300SpellerPerson}.}

\clearpage

To improve the spelling time many improvements and research has been done. Language models were used to predicted the word based on the first characters, which enabled great speedups\cite{LangModel}, classifiers were compared and tested on both healthy\cite{ClassTechniqueComp} and unhealthy subjects\cite{ComparisonClassifications}. Since many unhealthy subjects might have an impaired vision or eye movement, tactile\cite{TactileP300} and auditory\cite{AuditoryP300} spellers have been developed to circumvent this problem.

\npar

To improve accuracy, common problems such as adjacency distraction, when a subject is distracted from a neighboring flash, and double flashes, when the target row and column are flashed close after each other, were avoided using new randomized paradigms\cite{PaperThibault}. Other input layouts like the T9 interface P300 speller have also been developed\cite{P300T9}. 

\npar

To further speedup the spelling, error potentials were explored. Error potentials are triggered when the user becomes aware of en erroneous action\cite{ErrorPotentials}, i.e. when a wrong character is selected. When an Error potential is detected, the character is usually changed to the second most probable character according to the P300 decoding\cite{ErrSecChar}, which is the most viable character.

\subsection{Unsupervised P300 speller}

The basic P300 speller needs a calibration period before it can be used, when a healthy subject makes a mistake during calibration, he can communicate this and the problem can be resolved. This is not the case for a patient, who has no other means of communication than the P300 speller. Having wrongly labeled data during calibration can lead to severe problems. The unsupervised speller as proposed in \cite{P300Unsupervised} solves this problem by removing the need for a calibration procedure. The speller works with expectation maximization and has an undemanding linear classification backend and achieves good results, after a warm-up period when the system is adapting to the given condition.


\section{Benefits of creating an emotionally aware P300 speller}

Emotions play a major role in non-verbal communication, are quite complex and essential to understand human behavior. The ability to recognize emotions will improve the ability of computers to understand human interaction\cite{CompRecognizeEmotion} and are likely to improve the P300 speller's accuracy.

\subsection{Aid in Error potential detection}

To improve the detection of error potentials, emotion can be used. It is expected that when a person makes a lot of mistakes, his/her emotional state will change to a less happy, more frustrated state. Making the speller emotionally aware, could improve the detection of error potentials.

\clearpage

\subsection{Effect on the P300 wave}

Research with visual stimuli on healthy subjects, show that emotion has an effect on the auditory P300 wave\cite{AuditoryP300Effect}. Both the P300 peak amplitude and area was highest when viewing neutral pictures and descended further, in decreasing order, for sadness, anger and pleasure. The amplitudes were significantly lower at both Fz and C3 positions than Pz and Oz. The latency of the P300 ERP speller was shortest or neutrality and in increasing order longer for pleasure, anger and sadness. 

\npar

Interestingly, the P300 amplitudes were significantly larger for woman than for men. Additionally women showed lower amplitudes for for pleasure than neutrality or sadness while man showed smaller amplitudes for pleasure then either neutral or sadness. 

\subsection{Ethical issues}

Contrary to what subjects might think, the P300 speller is unable to read the mind and know what a person is thinking about\cite{P300Origin}. The P300 speller provides no more than a means of communication that the subject can use. Should he chose to ignore the instructions and focus his attention elsewhere, then the recordings become useless.

\npar

Knowing how the subject feels, can help him communicate more humane on one hand, while providing more insight for ethical issues, on the other hand, e.g. "How does the subject think about the P300 speller recording and analyzing his brain activity?". Information about the subject's emotional state can help answer some of these ethical questions.
