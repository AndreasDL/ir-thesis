\chapter{Introduction}
{\samenvatting This chapter describes the context of the thesis, starting with brain computer interfaces(BCI), before defining some BCI basics. After that, the P300 speller and P300 paradigm are introduced. Before the need for an emotionally aware P300 speller is justified, the basic process of emotion in the brain is explained. %TODO
}


\section{Brain computer interfaces}

A Brain Computer Interface (BCI)\nomenclature{BCI}{Brain Computer Interface}, creates a direct neural link from the brain to the computer\cite{LangModel}, that tries to recognize patterns and based on the extracted information, performs actions. A BCI removes to need for physical actions, i.e. typing or moving a mouse, for the transfer of information. The neural link provided by the BCI is made of two important components. The first component is the extraction component, which extract brain signals from the brain. The second component is the computer that interprets signals and performs actions based on the outcome.


\subsection{Electroencephalography (EEG)}
Different technologies exist to analyze the brain, the most convenient method is via Electroencephalography (EEG)\nomenclature{EEG}{Electroencephalography}, since it is a non-invasive method. Non-invasive methods, in contrast to invasive methods require no surgery; they simply measure electrical activity using electrodes placed on the scalp.

\npar

The electrical activity in a brain is caused when an incoming signal arrives in a neuron. This triggers some sodium ions to move inside the cell, which in turn, causes a voltage rise\cite{ExtendedPaper}. When this increase in voltage reaches a threshold, an action potential is triggered in the form of a wave of electrical discharge that travels to neighboring neurons. When this reaction occurs simultaneously in a lot of neurons, the change in electrical potential becomes significantly, making it visible to the EEG surface electrodes. EEG can thus only capture synchronized activity of many, many neurons.

\npar

Signals originating from the cortex, close to the skull, are most visible, while signals originating deeper in the brain cannot be observed directly. Even for signals originating close to the cortex, EEG is far from precise as the bone between the the cortex and electrodes distorts the signal. Additionally other artifacts like eye and muscle movement add a lot a noise to the signal, noise removal techniques are therefor advised. Even though the noise is persistent and EEG data has very low spatial resolution, it still can provide significant insight into the electrical activity of the cortex while offering excellent temporal resolution\cite{GivenPaper}.

\npar

Note that EEG records electrical activity, other methods like magnetoencephalography (MEG)\nomenclature{MEG}{magnetoencephalography} measure brain activity using magnetic fields. Since MEG is more prone to noise from external magnetic signals, i.e. the earth's magnetic field and electromagnetic communication, a magnetic shielded room is required, making this method very expensive and not mobile. 

\npar

EEG uses electrodes which are placed on the scalp to measure the electrical activity. To ensure that experiments are replicable, standards for locations of electrodes have been developed. One of these systems is the 10/20 system, an internationally recognized methods to describe location of scalp electrodes\cite{TenTwentyManual}. The numbers 10 and 20 refer to the distances between the electrodes, which are either 10\% or 20\% of the total front-back or left-right distance of the skull. Each site is identified with a letter that determines the lobe and hemisphere location.
\begin{itemize}
\item \textbf{F:} Frontal
\item \textbf{T:} Temporal
\item \textbf{C:} Central
\item \textbf{P:} Parietal
\item \textbf{O:} Occipital
\end{itemize}
Note that no central lobe exists; the C letter is only used for identification purposes. The letter z indicates that the electrode is placed on the central line. Even numbers are use for the right hemisphere, while odd numbers are used for the left hemisphere. A picture of a 23 channel 10/20 system is added below for clarification. Even though some experiment setups may use a different set of channels than shown in figure \ref{1020ElectrodePlacementSystem}, they all follow the same naming convention.

\mijnfiguur{width=0.9\textwidth}{1020ElectrodePlacementSystem}{The electrode placement of a 23 channel system\cite{1020Site}.}

Two different types of EEG channels exist, monopolar and dipolar. A monopolar channel records the potential difference of a signal, compared to a neutral electrode, usually connected to an ear lobe of mastoid. A bipolar channel is obtained by subtracting two monopolar EEG signals, which improves SNR by removing shared artifacts\cite{MonoBiPolar}. 

In the frequency domain, brain waves are usually split up into different bands\cite{EmotionRelativePower,WavesSite}, each band has a different medical interpretation. These wavebands \label{wavebands} are:
\begin{enumerate}
\item \textbf{Alpha:} 8-13Hz, indicate how relaxed and/or inactive the brain is.
\item \textbf{Beta:} 13-30HZ, indicate a more active and focused state of mind.
\item \textbf{Gamma:} 30-50Hz, relate to simultaneous processing of information from different brain areas.
\item \textbf{Delta:} 0-4hz, these waves are generated during dreamless sleep and meditation.
\item \textbf{theta:} 4-8Hz, occur during dreaming.
\end{enumerate}
Most muscle and eye artifacts have a frequency around 1.2Hz. Artificats caused by nearby power lines, have a frequency around 50Hz\cite{ExtendedPaper}. To remove most of this noise, a bandpass filter is usually applied to filter out frequencies below 4Hz and above 40-45Hz.



\section{Emotion recognition}

Psychology makes a clear distinction between physiological behavior and the conscious experience of an emotion, called expression\cite{ExtendedPaper}. The expression consists of many parts, including the facial expression, body language and voice concern. Unlike expression, the physiological aspect of an emotion, e.g. heart rate, skin conductance and pupil dilation, is much harder to control. To really know one's emotions, it seems, one has to research the physiological aspect of the emotion. One possibility for this is analysis of brain activity via Electroencephalography\cite{EEGDatasets}, which is the main method for this thesis.


\subsection{Emotion in the brain}
\label{valarrdomspace}

%\subsection{Arousal - valence (- dominance) classification}
Before emotions can be recognized, a classification model is needed. A common model to classify emotions is the bipolar arousal-valence model\cite{ExtendedPaper,RealTimeEEGEmotion}, that places emotions in a two dimensional space. The main advantage of using a multidimensional model, is that all emotions are modelled in its space, even when no particular discrete label can be used to define the current feeling. Figure \ref{ArousalValenceModel} shows the mapping of different emotions for this model. 

\npar
Even though arousal and valence describe emotion quite well, a third dimension can also be added. The new model then has three dimensions: arousal, valence and dominance. Arousal indicates how active a person is and ranges from inactive, bored to active, excited. The valence indicates if the emotion is perceived as positive or negative. The third dimension, the dominance, indicates how strong the emotional feeling was and ranges from a weak feeling to an empowered, overwhelming feeling. The dominance component can aid to filter out samples of strong feelings, since feelings with low dominance are less likely to show significant effects.

\mijnfiguur{width=0.45\textwidth}{ArousalValenceModel}{The arousal - valence model maps emotions in a two dimensional plane.}

\subsection{Features}
In this thesis two categories of features are observed. EEG features and non-EEG features. Non-EEG features are physiological signals like heart rate, skin conductivity, respiration rate, ... They are easy to measure as they do not require an EEG cap. The EEG features are features extracted from the EEG signals of a subjects. Note that the literature does not agree on a specific set of EEG signals nor does it agree on what channels and/or waveband are important. However the literature does agree on certain things; the right hemisphere of a subject is generally speaking, more active during negative emotions than the left hemisphere, which is in turn more active during positive emotions\cite{RealTimeEEGEmotion,EEGDatasets,killyPaper}. The features are discussed in more depth in \ref{featuresExplained}.

%machine learning
\section{Machine learning}
The next important topic that needs to be covered is machine learning. Machine learning is the missing link between the features and the emotion recognition. As machine learning is a very broad domain, the discussion will be limited to the application of machine learning and machine learning techniques as this is the most relevant part for this thesis. One possible definition for Machine learning is: "the science of getting computers to act without being explicitly programmed". To do so, machine learning uses pattern recognition to find patterns or structure in the data. A simple example of machine learning is the Optical Character Recognition (OCR)\nomenclature{OCR}{Optical Character Recognition}, where a computer recognises characters in pictures.

\npar

Let's get a look at the following example, to further explain how machine learning works. Suppose one has a price list of houses that are for sale combined with their total area. Logic sense dictates us that a bigger house will have a higher asking price than a smaller house. Therefore the asking price of a house is correlated to the asking price. Suppose you want to predict how much a certain home is worth, based on their area. This is possible with machine learning, first you need to train your machine learning algorithm with a list of asking prices and the corresponding area of the house. This should give you a coefficient, lets say you pay 1000 euro for each square meter. Once this is done you can predict prices of new houses based on the corresponding area. 
\npar
This will give some reasonable results, but the algorithm will probably have some flaws. This is due to the fact that the area of the house is only one feature that determines the price, there are many other that we haven't taken into consideration. Looking with more detail at the data, i.e. adding additional features will thus improve the performance of our algorithm. For example, a house with 5 bedrooms is more expensive than a house with only 3 bedrooms.

\npar

Machine learning algorithms are responsible for finding the relation between features and the predicted value. There exist many machine learning algorithms, one way to group these algorithms, is to look at their produced output. In the asking price examples above, the output is a price, which is a continuous value. The OCR example from above, where characters are recognized in a picture is a classification problem, as there is only a limited set of characters. 

\npar

Another way to group the algorithms is based on their training data. In the asking price examples above one gets labelled results; the asking price is given for each area, this is referred to as supervised machine learning. The other possibility is unsupervised machine learning, which often results in finding groups of similar data points (clustering), without knowing the actual labels. Note that the combination of supervised and unsupervised data, also known as semi supervised learning, is also possible. Suppose you have a dataset with 5000 webpages and you want to categorise them in 10 distinct categories, e.g. science, nature, cooking, ... , but you only have the labels for 100 of the 5000 pages. Then you could first cluster the pages in similar groups using unsupervised learning. As soon as a group contains a labelled page, you can label all the pages in the group, since clustering returns groups of similar examples. Semi supervised learning has the advantage that one can also use unlabelled data. Unlabelled data is often easy and cheaper to obtain, unlike labelled data which is usually quite rare; if you had a fast and easy way to label the data then you wouldn't be needing machine learning.

\npar

In this thesis machine learning is used to find patterns in the aforementioned features that indicate the user's emotional state. The general process of machine learning is shown in Figure \ref{eegtopred}, the process starts with gathering EEG data, from which features are selected. These features are then fed to a machine learning algorithm, which outputs a prediction.

\mijnfiguur{width=0.7\textwidth}{eegtopred}{Basic steps of machine learning.}


\subsection{Over and underfitting / high bias and high variance}

\mijnfiguur{width=0.9\textwidth}{overunderfitting}{Overfitting versus underfitting\cite{overunderfitting}.}

Suppose the example in Figure \ref{overunderfitting}, where one tries to find a good function to fit the given data points. Looking at the three proposed functions, one can easily see that the middle figure is the most likely generator function of the red points. 

The figure on the left corresponds to an underfit, where the proposed function is not able to capture sufficient detail of the points. The function is not complex enough to approach the generator function, which is known as high bias. A high bias problem has a high training error, as the function is not able to fit the points correctly, this is visible in Figure \ref{highbias}

\mijnfiguur{width=0.6\textwidth}{highbias}{A high bias function is not complex enough to approach the generator function closely.}

The function on the rights The function on the right corresponds to an overfit; the function 'goes through' each point exactly, but one can see that in between data points the behaviour of the hypothesis function is not logical. This problem is known as a high variance problem, where the train error is close to zero, but the test error is quite dramatic. 

\mijnfiguur{width=0.6\textwidth}{highvariance}{A High variance function is too complex and fits the data point too closely.}

Another way to explain the bias variance tradeoff is by an example. Suppose you have a dart board, as shown in Figure \ref{BiasVariance}. Suppose the situation on the top left corner, this corresponds to a world class player that has perfect aim, and very little variation on his precision. The situation on the left bottom corresponds to a player that has very little variation on his precision, but that is consistently aiming too high, he is biased to hit higher than needed. The pictures on the right side are different, there the person may or may not have a biased aim, but it is clear that he has a lot of variation in the precision of his aim.
\npar
In the context of machine learning, the low bias corresponds to having a hypothesis set that is close to the generator function, which allows you to get quite close. However you still have to pick the right function from that set, which is hard to do if you don't have enough data. If you are not able to take the best solution from the hypothesis set, you have high variance.

\mijnfiguur{width=0.6\textwidth}{BiasVariance}{The bias variance explained using the dartboard example found at \cite{biasvariance}}

%goal of thesis
\section{Goal of the thesis}
The first goal is finding relevant features for emotion recognition in a person specific setting. This algorithm tries to accurately predict the emotions of one person, using only training data from that person. The Second goal is finding features for emotion recognition in a cross-persons setting. In this setting the features should generalise well across different persons, thus the algorithm should be able to recognize emotions from unseen persons.

\npar

The main problem is that there are already a lot of features known, but, as is often the case with EEG data, training data is expensive and limited. Using a lot of features will thus quickly result in overfitting. Additionally, a lot of different features are reported in the literature, as you can see in Table \ref{diffFeat}.

\begin{table}[]
\centering
\caption{Six different papers on emotion recognition, six different feature sets}
\label{diffFeat}
\begin{tabular}{ll}
\textbf{study} & \textbf{features used}                         \\
\citep{ref4}     & Alpha and beta power                           \\
\citep{ref7}     & PSD and asymmetry features                     \\
\citep{ref8}     & PSD                                            \\
\citep{ref6}     & discrete wavelet transform of alpha, beta and gamma band \\
\citep{ExtendedPaper}	&	alpha/beta ratio, Fpz beta and alpha band power \\
\citep{killyPaper} & PSD, RCAU, DCAU, DASM, RASM, DE \\
\end{tabular}
\end{table}

\npar %TODO person specific ref

Another point to note is that even though a simple limited set of features might work, it is more likely to have less accurate results as optimal features might have been left out. This problem is even more severe when considering that EEG data is person specific, features that work good for one person, might not work for another person. Finding a good set of features that works for all persons is a non trivial problem. One solution to this problem could be to use a large pool of possible features from which a limited set of good features are selected. This allows to provide good features to the machine learning algorithm, while still keeping the set of features limited in size. The machine learning algorithm which features to use and which to neglect.

%data
\subsection{Dataset}
One of the most used datasets in the context of emotion recognition is the Dataset for Emotion Analysis using Physiological Signals (DEAP)\nomenclature{DEAP}{Dataset for Emotion Analysis using Physiological Signals} dataset\cite{DEAP}.

\npar

This dataset consists of several parts, the first part is a rating of 120 music videos by 14 - 16 persons. Each video is rated for valence, arousal and dominance on a scale ranging from 1 to 9. This part of the dataset is not used during this thesis, because it contains no EEG recordings.

\npar

The next part of the dataset is the physiological experiment that contains emotional reactions of 32 subjects. The emotional reactions were triggered using music video excerpts; each subject watched 40 one-minute videos, while several physiological signals were recorded. These physiological signals consist of 32 channel 512Hz EEH and peripheral physiological signals. Additionally 22 of the 32 subjects were recorded, so research in facial expressions is also possible with this dataset. These videos are also rated for arousal, valence and dominance. Additionally, the video were rated for liking. It is important to note that liking is not the same as valence, e.g. a user can like a sad song. As liking is no dimension of the emotional space, it is neglected for this thesis.

\npar

More concretely, this dataset contains following signals:
\begin{table}[]
\centering
\caption{The available signals in the DEAP dataset}
\label{DEAPSignals}
\begin{tabular}{lll|lll}
\textbf{channel} & \textbf{name} & \textbf{category} & \textbf{channel} & \textbf{name}    & \textbf{category} \\
\textbf{1}       & Fp1           & EEG               & \textbf{21}      & F8               & EEG               \\
\textbf{2}       & AF3           & EEG               & \textbf{22}      & FC6              & EEG               \\
\textbf{3}       & F3            & EEG               & \textbf{23}      & FC2              & EEG               \\
\textbf{4}       & F7            & EEG               & \textbf{24}      & Cz               & EEG               \\
\textbf{5}       & FC5           & EEG               & \textbf{25}      & C4               & EEG               \\
\textbf{6}       & FC1           & EEG               & \textbf{26}      & T8               & EEG               \\
\textbf{7}       & C3            & EEG               & \textbf{27}      & CP6              & EEG               \\
\textbf{8}       & T7            & EEG               & \textbf{28}      & CP2              & EEG               \\
\textbf{9}       & CP5           & EEG               & \textbf{29}      & P4               & EEG               \\
\textbf{10}      & CP1           & EEG               & \textbf{30}      & P8               & EEG               \\
\textbf{11}      & P3            & EEG               & \textbf{31}      & PO4              & EEG               \\
\textbf{12}      & P7            & EEG               & \textbf{32}      & O2               & EEG               \\
\textbf{13}      & PO3           & EEG               & \textbf{33}      & hEOG             & non-EEG           \\
\textbf{14}      & O1            & EEG               & \textbf{34}      & vEOG             & non-EEG           \\
\textbf{15}      & Oz            & EEG               & \textbf{35}      & zEMG             & non-EEG           \\
\textbf{16}      & Pz            & EEG               & \textbf{36}      & tEMG             & non-EEG           \\
\textbf{17}      & Fp2           & EEG               & \textbf{37}      & GSR              & non-EEG           \\
\textbf{18}      & AF4           & EEG               & \textbf{38}      & respiration belt & non-EEG           \\
\textbf{19}      & Fz            & EEG               & \textbf{39}      & plethysmograph   & non-EEG           \\
\textbf{20}      & F4            & EEG               & \textbf{40}      & temperature      & non-EEG          
\end{tabular}
\end{table}


\npar

There also exists a preprocessed version of the physiological experiment database, which has downsampling to 128Hz, noise removal and EOG artifact removal. This dataset is used during this thesis.
